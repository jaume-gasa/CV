%%%%%%%%%%%%%%%%%%%%%%%%%%%%%%%%%%%%%%%%%
% "ModernCV" CV and Cover Letter
% LaTeX Template
% Version 1.1 (9/12/12)
%
% This template has been downloaded from:
% http://www.LaTeXTemplates.com
%
% Original author:
% Xavier Danaux (xdanaux@gmail.com)
%
% License:
% CC BY-NC-SA 3.0 (http://creativecommons.org/licenses/by-nc-sa/3.0/)
%
% Important note:
% This template requires the moderncv.cls and .sty files to be in the same
% directory as this .tex file. These files provide the resume style and themes
% used for structuring the document.
%
%%%%%%%%%%%%%%%%%%%%%%%%%%%%%%%%%%%%%%%%%

%----------------------------------------------------------------------------------------
%	PACKAGES AND OTHER DOCUMENT CONFIGURATIONS
%----------------------------------------------------------------------------------------

\documentclass[11pt,a4paper, sans]{moderncv} % Font sizes: 10, 11, or 12; paper sizes: a4paper, letterpaper, a5paper, legalpaper, executivepaper or landscape; font families: sans or roman



\moderncvstyle{classic} % CV theme - options include: 'casual' (default), 'classic', 'oldstyle' and 'banking'
\moderncvcolor{blue} % CV color - options include: 'blue' (default), 'orange', 'green', 'red', 'purple', 'grey' and 'black'

\usepackage{lipsum} % Used for inserting dummy 'Lorem ipsum' text into the template
\usepackage[utf8]{inputenc}

%\usepackage[scale=0.75]{geometry} % Reduce document margins
\usepackage[top=0.8in, bottom=1in, left=1in, right=1.25in]{geometry} % Reduce document margins


%\setlength{\hintscolumnwidth}{3.25cm} % Uncomment to change the width of the dates column
%\setlength{\makecvtitlenamewidth}{10cm} % For the 'classic' style, uncomment to adjust the width of the space allocated to your name

%----------------------------------------------------------------------------------------
%	NAME AND CONTACT INFORMATION SECTION
%----------------------------------------------------------------------------------------

\firstname{Jaume} % Your first name
\familyname{Gasa Gómez} % Your last name

% All information in this block is optional, comment out any lines you don't need
\title{Inteligencia Artificial y Robótica}

%\address{San Vicente del Raspeig}{Alicante 03690}
%\phone{605 871 039}
%\email{ua.jaume@gmail.com}

\address{XXX}{XXX}
\phone{XXX}
\email{XXX}



\homepage{github.com/jaume-gasa} % The first argument is the url for the clickable link, the second argument is the url displayed in the template - this allows special characters to be displayed such as the tilde in this example
%\extrainfo{additional information}
%\photo[70pt][0.4pt]{picture} % The first bracket is the picture height, the second is the thickness of the frame around the picture (0pt for no frame)
%\quote{"A witty and playful quotation" - John Smith}

%----------------------------------------------------------------------------------------

\begin{document}

\makecvtitle % Print the CV title

%\section{Datos personales}
%\cvitem{Fecha de nacimiento}{14 de Julio de 1992}
%\cvitem{Lugar de nacimiento}{Barcelona, España}
%\cvitem{DNI}{53245501H}
%\cvitem{Domicilio}{San Vicente del Raspeig, Alicante, España}
%\cvitem{Email}{ua.jaume@gmail.com}
%\cvitem{Teléfono}{605 871 039}

\section{Formación}
\cvitem{2017--}{\textbf{Máster en Automática y Robótica}, \textit{Universidad de Alicante}.}
\cvitem{2016}{\textbf{Grado en Ingenieria Informática}, \textit{Universidad de Alicante}.}
\cvitem{2015}{\textbf{Introduction to Computational Thinking and Data Science}, \textit{Massachusetts
Institute of Technology through edX}.}
\cvitem{2014}{\textbf{Curso CECLEC impresoras REPRAP}, \textit{Universidad de Alicante}.}

\section{Conocimientos}
\renewcommand{\listitemsymbol}{-~} % Changes the symbol used for lists

\cvitem{}{\textbf{Python} avanzado.}
\cvitem{}{Cómodo en lenguajes como C/C++, Java y sistemas embebidos (Arduino y Raspberry Pi).}
% \cvlistitem{Programas de diseño 3D: OpenSCAD, Blender}

\cvitem{}{Librerías de inteligencia artificial y ciencia de datos: NumPy, Matplotlib, Scikit-learn, Keras, Tensorflow y Lasagne(Theano).}
\cvitem{}{Conocimientos avanzados de sistemas GNU/Linux (x86-64 y ARM).}
\cvitem{}{Herramientas de impresión y diseño 3D: Pronterface, Cura y OpenSCAD.}
\cvitem{}{Composición de documentos \LaTeX{}.}

\section{Experiencia}
\cvitem{2016-}{\textbf{Investigador}, \textit{Departamento Visión Robótica}, Universidad de Alicante.}
\cvitem{}{Desarrollo de un coche autónomo con Raspberry Pi utilizando redes neuronales recurrentes.}
\cvitem{2016--}{\textbf{Adminstrador de sistemas}, \textit{ANDA Alicante}.}
\cvitem{}{Configuración y mantenimiento de equipos informáticos.}
 

\section{Eventos}
\cvitem{2017}{\textbf{Ponente en Mes Cultural EPS}, \textit{Introducción al deep learning con Keras}. \href{https://github.com/jaume-gasa/IntroduccionDeepLearningMesCultural2017ALC/blob/master/presentacion.pdf}{[Diapositivas]}}
\cvitem{2016}{\textbf{Ponente en PyDay Alicante}, \textit{Estado del arte del deep learning}.\href{https://github.com/jaume-gasa/EstadoDelArteDeepLearningPyDayAlicante/blob/master/presentacion.pdf}{ [Diapositivas]}}


\section{Proyectos}
\cvitem{2016--}{Control de prótesis 3D mioeléctrica mediante dispositivos EMG (MYO armband), visión (Leap Motion) y redes neuronales.}
\cvitem{2014}{Web de traducción de texto castellano a braille y generación de un modelo para su impresión 3D.} 

%\cvitem{}{Control de prótesis 3D mioeléctrica: uso de OpenCV y un dispositvo EMG para obtener datos del movimiento
%de los dedos de una mano y usarlos en una red neuronal, desarrollada en Lasagne, para
%controlar los dedos de una prótesis hecha mediante impresión 3D.}

%\cvitem{}{Traducción de texto a braille y generación de un modelo para su impresión 3D: puesta en marcha desde cero %de una web escrita en Django capaz de traducir texto en castellano y braile para posteriormente, mediante un
%\textit{script} en OpenSCAD generar un modelo en 3D del texto en braille con sus medidas estándar.}

\section{Idiomas}
\cvitem{2015}{Acreditación CertACLES B1 inglés, \textit{Universidad de Alicante}.}
\cvitem{2014}{Inglés B2 oral, curso inglés intensivo, \textit{Universidad Menendez Pelayo}.}

\end{document}
